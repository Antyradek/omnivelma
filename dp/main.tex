\documentclass{beamer}
\usepackage{amsmath}
\usepackage{icomma}
\usepackage[utf8]{inputenc}
\usepackage[T1]{fontenc}
\usepackage{polski}
\usepackage[polish]{babel}
\usepackage{hyperref}
\usepackage{float}
\usetheme{Darmstadt}
\usecolortheme{rose}

% Naprawa nazw z angielskiego
\def\figureautorefname{Rysunek}

%Strona streszczenia
\newenvironment{abstractpage}
  {\cleardoublepage\vspace*{\fill}\thispagestyle{empty}}
  {\vfill\cleardoublepage}
  
%Samo streszczenie
\newenvironment{abstractsection}[1]
  {\bigskip\selectlanguage{#1}%
   \begin{center}\bfseries\abstractname\end{center}}
  {\par\bigskip}

%Ładne ułamki w jednostkach fizycznych
\sisetup{per-mode=symbol}%

\beamertemplatenavigationsymbolsempty
\setbeamertemplate{footline}[frame number]

\begin{document}
  \section{Wprowadzenie}
  \begin{frame}
    \title[Omnivelma]{Symulacja dookólnej bazy mobilnej}
	\author{Radosław Świątkiewicz}
	\date{\today}
	\institute{Wydział Elektroniki i Technik Informacyjnych \\ Politechnika Warszawska}
	\titlepage
  \end{frame}
  \begin{frame}
	\frametitle{Spis treści}
	\tableofcontents[currentsection]
  \end{frame}
  \begin{frame}
	\begin{description}
	 \item[Autor] Radosław Świątkiewicz
	 \item[Promotor] dr hab. inż. Wojciech Szynkiewicz \\ Zespół Programowania Robotów i Systemów Rozpoznających \\ Instytut Automatyki i Informatyki Stosowanej
	\end{description}
  \end{frame}
  
  \section{Cel}
  \begin{frame}
	\frametitle{Platforma dookólna}
	Zdjęcia platformy i Velmy
  \end{frame}
  \begin{frame}
	\frametitle{Po co jest potrzebny model}
	żeby fajnie się robiło
  \end{frame}



\end{document}