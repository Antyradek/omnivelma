\chapter{Implementacja} 
\label{sec:implementation}
W tym rozdziale opisane są szczegóły techniczne zastosowanych rozwiązań.

\section{Istniejące implementacje}
	Istnieją także inne modele jeżdżących robotów na kołach szwedzkich.
	Można z nich brać przykład i sugerować się źródłami kodu i budową modeli.

	Kuka Youbot jest popularnym robotem wielokierunkowym. Jego modele są domyślnie dostępne w różnych symulatorach, między innymi w Gazebo i V-Repie, które są dobrymi kandydatami do 
	użycia w projekcie.
	Tylko w przypadku V-Repa, istnieje wstępny sterownik do którego da się wysyłać odpowiednie wartości kierunku, a on nadaje takie prędkości kołom, aby poruszać modelem w zadanym kierunku.
	Wersja dla Gazebo jest statycznym obiektem z błędnie ustanowionymi przegubami, jego efektory nie są zaimplementowane.
	
	Dodatkowo, V-Rep posiada wbudowane dwa inne pojazdy o napędach kół Mecanum i czujnikach laserowych.
	Zewnętrzne modele także pomogą przy wstępnej weryfikacji zachowania się budowanego tutaj modelu, czy nie zachowuje się nadzwyczaj dziwnie w pierwszych fazach projektu.

	Ze względu na niezwykle zaawansowany obiekt kół i kształt rolek, ważne jest aby uprościć model, poprzez zamianę niektórych składowych i dodanie sztucznych więzów.
	Całościowy model może być zbyt skomplikowany, aby maszyny symulacji mogły go obliczać w czasie rzeczywistym.
	Dokładny model także jest znacznie trudniej poprawnie wymodelować, ze względu na liczne tarcia i poślizgi rolek.
	Proponowane uproszczenia modeli opisane są w sekcji \ref{sec:omnivelma}.
	

\section{Ogólne typy implementacji}
	\subsection{Program wykonywalny w ROS}

	\subsection{Wtyczka Gazebo}
		\subsubsection{Wtyczka sterownika modelu}
		\subsubsection{Wtyczka sterownika czujnika}
	
\section{Program ręcznego sterowania}
%TODO diagram klas


%TODO implementacja wszystkiego

% 
% 	Skrypt kompilacji ROSa, zapisany w CMake, dba o odpowiednie podawanie ścieżek, kolejność kompilacji pakietów i załączanie nazw.
% 	Na przykład, jeśli pakiet wymaga pliku nagłówkowego, generowanego przez kompilację innego pakietu, załącza go w kodzie tak, jakby był systemowy.
% 	Skrypt CMake zadba o wywołanie kompilatora z odpowiednimi argumentami.

%DO ROSA
Instalacja programu na systemie operacyjnym jest złożona.
	Z wyjątkiem odpowiednich wersji Ubuntu, nie ma łatwego sposobu na instalację go na innych systemach.
	Na przeszkodzie stoją błędy kompilacji dla nowszych wersji kompilatorów, zależności od dokładnych wersji zewnętrznych bibliotek i 
	inne problemy w czasie wykonywania, jak naruszenie ochrony pamięci. 
	Instalacja alternatywnych pakietów i ręczna kompilacja niektórych części nie działa we wszystkich przypadkach.

	Rozwiązaniem tego problemu jest instalacja tej platformy programistycznej na maszynie wirtualnej, lub na systemie uruchamianym z dysku zewnętrznego. 
	Najnowszą wersją ROSa jest \emph{Lunar Loggerhead} z maja 2017, jednak nie jest to wersja długiego wsparcia, a co za tym idzie, nie posiada wszystkich
	pakietów zewnętrznych twórców, potrzebnych przy wizualizacji symulacji.
	Odpowiedniejszą wersją jest \emph{Kinetic Kame} z marca 2016 roku, o bardzo dobrym wsparciu.
	Pakiety składające się na system ROS nadal są regularnie aktualizowane, lecz nie zawierają nowych funkcjonalności, a jedynie poprawki błędów.
	Główny symulator fizyki, najważniejszy program, jest w tej samej wersji w obu dystrybucjach.

	Uruchomienie platformy programistycznej na systemie wymaga wielu dodatkowych komend inicjalizujących, 
	a także dopisywania do tworzonych projektów licznych plików konfiguracyjnych za pomocą dostarczonych skryptów.
	Używanie modułów z linii poleceń wymaga ustawienia kilku zmiennych systemowych, poprzez wczytywanie skryptów.
	Użycie niektórych funkcji ROS wymaga uruchomionego demona serwera w tle.

	Ogólnie instalacja i używanie ROS na systemie zostawia dużo różnorodnych plików w katalogu domowym, co może nie być wskazane na codziennym systemie operacyjnym.
	Z drugiej jednak strony, wirtualizacja systemu operacyjnego z ROS bardzo ogranicza dostępną moc obliczeniową, potrzebną takim programom w dużych ilościach.
