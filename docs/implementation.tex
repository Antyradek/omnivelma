\chapter{Implementacja} 
\label{sec:implementation}
W tym rozdziale opisane są szczegóły techniczne zastosowanych rozwiązań.

\section{Istniejące implementacje}
	Istnieją także inne modele jeżdżących robotów na kołach szwedzkich.
	Można z nich brać przykład i sugerować się źródłami kodu i budową modeli.

	Kuka Youbot jest popularnym robotem wielokierunkowym. Jego modele są domyślnie dostępne w różnych symulatorach, między innymi w Gazebo i V-Repie, które są dobrymi kandydatami do 
	użycia w projekcie.
	Tylko w przypadku V-Repa, istnieje wstępny sterownik do którego da się wysyłać odpowiednie wartości kierunku, a on nadaje takie prędkości kołom, aby poruszać modelem w zadanym kierunku.
	Wersja dla Gazebo jest statycznym obiektem z błędnie ustanowionymi przegubami, jego efektory nie są zaimplementowane.
	
	Dodatkowo, V-Rep posiada wbudowane dwa inne pojazdy o napędach kół Mecanum i czujnikach laserowych.
	Zewnętrzne modele także pomogą przy wstępnej weryfikacji zachowania się budowanego tutaj modelu, czy nie zachowuje się nadzwyczaj dziwnie w pierwszych fazach projektu.

	Ze względu na niezwykle zaawansowany obiekt kół i kształt rolek, ważne jest aby uprościć model, poprzez zamianę niektórych składowych i dodanie sztucznych więzów.
	Całościowy model może być zbyt skomplikowany, aby maszyny symulacji mogły go obliczać w czasie rzeczywistym.
	Dokładny model także jest znacznie trudniej poprawnie wymodelować, ze względu na liczne tarcia i poślizgi rolek.
	Proponowane uproszczenia modeli opisane są w sekcji \ref{sec:omnivelma}.
	

\section{Ogólne typy implementacji}
	\subsection{Program wykonywalny w ROS}

	\subsection{Wtyczka Gazebo}
		\subsubsection{Wtyczka sterownika modelu}
		\subsubsection{Wtyczka sterownika czujnika}
	
\section{Program ręcznego sterowania}
%TODO diagram klas


%TODO implementacja wszystkiego
