\chapter{Opis problemu}
\section{Cel}
Celem tej pracy jest budowa i zaprogramowanie symulacji dla fizycznej podstawy jezdnej.

Używając do tego środowiska symulacyjnego należy stworzyć model i zapisać do niego API, aby mogło być z zewnątrz sterowane poprzez inną bibliotekę.
Oddzielając model od sterowania jesteśmy potem łatwo zamienić model na prawdziwego robota zachowując to samo sterowanie.

\section{Podstawa jezdna}
Jest to duża podstawa jeżdżąca na czterech kołach szwedzkich, co robi z niej pojazd omnikierunkowy.
Każde koło jest sterowane osobno przez serwomotoryczny silnik elektryczny.

Wielkość i nośność podstawy jest wystarczająco duża, aby utrzymać wydziałowego robota manipulującego Velma.
Jeżdżąc na tej podstawie może on przemieszczać się i obracać w dowolnym kierunku, aby uzyskać lepszy dostęp do manipulowanych przedmiotów.

% zawias

% czym zajmują się odpowiednie składniki

% technologie - Czemu Gazebo, V-rep i ROS
% Gazebo przydatnie, otwarty opis robota, natywny ROS, C++ i czysty kod
% V-Rep dziwny opis, nietypowe i skomplikowane funkacje w czytym C, Lua?


% uruchomienie - Czemu Ubuntu i co w Ubuntu

% sposób modelowania (w edytorze tekstowym)

% sposób pisania oprogramowania

% Wymagania oprogramowania na niezawodność

% istniejące implemetacje (Yuka w Gazebo, co jest tylko modelem)
% Yuka w V-Rep, która działa w pełni
