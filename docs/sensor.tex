\chapter{Model czujnika laserowego}
Istnieje dedykowany obiekt w standardzie SDF dla tego typu czujników.
Również Gazebo wspiera wizualnie symulację poprzez możliwość renderowania obliczonych półprostych.

\section{Obliczenia symulatora}
Czujnik laserowy jest bardzo łatwo zasymulować w przestrzeni wirtualnej za pomocą rzutowania półprostych.
Jest to jedna z podstawowych technik renderowania obrazu.
Używa się jej także przy obliczaniu symulacji fizycznej i specjalnych wydarzeń związanych na przykład z grami komputerowymi.

Półprosta jest emitowana z jakiegoś punktu w jakimś kierunku w przestrzeni trójwymiarowej.
Następnie system próbuje znaleźć pierwszy punkt jej kolizji z jakimś symulowanym ciałem fizycznym, posiadającym odpowiedni kolider 
Ponieważ zasoby komputera zawsze są ograniczone, symulacja półprostej także musi mieć pewien limit. 
Zwykle jest on jednak na tyle duży, że z punktu widzenia lokalnych wydarzeń, można uznać tą odległość za nieskończoną.

Algorytm obliczania kolizji z półprostą bazuje na kosztowym porównywaniu pozycji każdego obiektu fizycznego na scenie.
Istnieją oczywiście sposoby na zmniejszenie ilości obliczeń, na przykład metoda prostopadłościanów zawierających obiekt, ale sposób radzenia sobie z tym nie jest
częścią tematu pracy,
Wystarczy wspomnieć, że symulacja dużej ilości laserów jest operacją kosztowną.

\section{Różnice między czujnikiem, a modelem}
Półprosta emitowana jest z puntu reprezentującego środek czujnika, naturalnie, model upraszcza rzeczywisty czujnik (budowa czujnika laserowego została opisana w sekcji \ref{lidar}).
Uproszczenie polega na tym, że nie ma w środku żadnego lustra lub obracającej się części. 
W rzeczywistości w czujniku jest jeden laser, emitujący pulsy w określonych odstępach czasu.
W modelu można zatem przyjąć osobne półproste dla każdego pulsu lasera.

Można zauważyć tym samym, że model wydaje się lepszym czujnikiem, niż rzeczywisty LiDAR.
W danej chwili model emituje promień we wszystkich kierunkach jednocześnie, podczas gdy czujnik jednym pulsem może dokonać tylko jednego pomiaru o danym w tej chwili kącie.
Jednakże dyskretny sposób symulacji powoduje, że w obu przypadkach dane są podawane w grupach.
Czujnik jest wstanie wysłać pakiet z danymi ostatniego pomiaru, podczas gdy program modelujący czujnik jest obsługiwany na zasadzie przerwań czasowych 
po każdej klatce i tylko wtedy może wywołać funkcje zwracające dane zasymulowanych pomiarów.
To oznacza, że interfejs do ich obsługi wcale nie jest aż tak inny.
Dodatkowo, w zależności od obciążenia komputera, model czujnika jest podatny na opóźnienia w odczytywaniu stanu.

\section{Komunikacja}
Jednakże, bazując na architekturze opisanej wcześniej na rysunku \ref{fig:agent}, należy tak zbudować system, aby program komunikował się w identyczny sposób z 
modelem czujnika, jak i samym czujnikiem.
Służą do tego specjalne pakiety ROSa, zawierające czas pomiaru, typ i dane.
Program obsługujący model czujnika generuje i wysyła pakiety typu \texttt{sensor\_msgs/LaserScan}.

Identycznie, inny program, podłączony do czujnika za pomocą jednego z interfejsów, także powinien generować takie same pakiety.

\section{Model w Gazebo}
Podobnie, jak w modelu platformy fizycznej, należy napisać SDF...
