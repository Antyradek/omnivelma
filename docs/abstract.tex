 \begin{abstractpage}
	\begin{abstractsection}{polish}
		Praca opisuje projektowanie i budowę środowiska symulacyjnego dla wielokierunkowej platformy mobilnej, poruszającej się za pomocą kół szwedzkich.
		Platforma jest bazą mobilną dla dwuramiennego robota Velma. 
		
		Celem pracy jest stworzenie możliwie dokładnego modelu symulacyjnego rzeczywistej bazy mobilnej.
		Model ten będzie służyć do wstępnych badań algorytmów planowania ruchu i sterowania robotem mobilnym.

		Rozpatrzone są tutaj wymagania i problemy przy tworzeniu każdego ze składników środowiska.
		Na system składają się wirtualne efektory i receptory obsługujące odpowiednią maszynę symulacyjną, a także pakiety wspomagające symulację.
		Do wykonania zadania użyto programowej struktury ramowej ROS i symulatora Gazebo.
		
		W ramach pracy, stworzone zostały modele dynamiczne, oraz kinematyczne platformy.
		Stworzono modele skanerów laserowych, jednostki inercyjnej, oraz czujnika enkoderów, a także narzędzia wspomagające testowanie i uruchomienie odpowiednich składników systemu.
	\end{abstractsection}

	\begin{abstractsection}{english}
		This paper describes creation of simulation environment for omnidirectional platform.
		Robot, which is powered by four Mecanum wheels, is a mobile base for larger robotic manipulator.
		
		The platform model and it's sensors models will then support the development of path-planning and movement algorithms, in order to safely test them before running on
		hardware.
		
		To execute the task, ROS framework and Gazebo simulator were used.
		Simulation environment consists of dynamic and kinematic models, laser scanner, inertial measurement unit and simulated encoder. 
		Also, many other components have been created in order to ease testing, data logging and execution of simulation.
	\end{abstractsection}
\end{abstractpage}
