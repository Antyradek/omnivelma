 \begin{abstractpage}
	\begin{abstractsection}{polish}
		Praca opisuje projektowanie i budowę środowiska symulacyjnego dla wielokierunkowej platformy mobilnej, poruszającej się za pomocą kół szwedzkich.
		Platforma jest bazą mobilną dla dwuramiennego robota Velma. 
		
		Celem pracy jest stworzenie możliwie dokładnego modelu symulacyjnego rzeczywistej bazy mobilnej.
		Model ten będzie służyć do wstępnych badań algorytmów planowania ruchu i sterowania robotem mobilnym.

		Rozpatrzone są tutaj wymagania i problemy przy tworzeniu każdego ze składników środowiska.
		Na system składają się wirtualne efektory i receptory obsługujące odpowiednią maszynę symulacyjną, a także komponenty wspomagające symulację.
		Do wykonania zadania użyto programowej struktury ramowej ROS i symulatora Gazebo.
		
		W ramach pracy stworzone zostały modele dynamiczne, oraz kinematyczne platformy.
		Stworzono modele czujników laserowych, czujnika inercji, oraz czujnika enkoderów, a także narzędzia wspomagające testowanie i uruchomienie odpowiednich składników systemu.
	\end{abstractsection}

	\begin{abstractsection}{english}
		%This paper describes creation of simulation environment for omnidirectional platform.
		
		%The model of the platform will then be used in testing of motion-planning algorithms to later be used on robot.
	
		[TODO]
		
	\end{abstractsection}
\end{abstractpage}
