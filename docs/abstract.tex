\begin{abstract}
Ta praca opisuje projektowanie i budowę środowiska symulacyjnego dla wielokierunkowej platformy mobilnej poruszającej się za pomocą kół szwedzkich.
Platforma i robot, którego ma wozić są własnością wydziału Elektroniki i Technik Informacyjnych na Politechnice Warszawskiej.
Celem jest przygotowanie jak najdokładniejszej kopii oryginału, aby użyć jej zewnętrznym w programie sterującym bez jego modyfikacji.

Rozpatrzone są tutaj wymagania i problemy przy tworzeniu każdego ze składników środowiska.
Na system składają się wirtualne efektory i receptory obsługujące odpowiednią maszynę symulacyjną.
\end{abstract}
 
