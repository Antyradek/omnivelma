{\centering\Large\bfseries Streszczenie \\}
\noindent{\bfseries Tytuł: } Symulacja dookólnej bazy mobilnej \\
\noindent{\bfseries Słowa kluczowe: } \texttt{symulacja}, \texttt{Gazebo}, \texttt{ROS}, \texttt{koła Mecanum}, \texttt{wielokierunkowa platforma mobilna}, \texttt{robot Velma}, \texttt{model dynamiki}, \texttt{ODE}.

Niniejsza praca dotyczy projektowania i budowy środowiska symulacyjnego dla wielokierunkowej platformy mobilnej, poruszającej się za pomocą kół szwedzkich.
Platforma jest bazą mobilną dla dwuramiennego robota Velma. 

Celem pracy jest stworzenie możliwie dokładnego modelu symulacyjnego rzeczywistej bazy mobilnej.
Model ten będzie służyć do wstępnych badań algorytmów planowania ruchu i sterowania robotem mobilnym.

Rozpatrzone są wymagania i problemy przy tworzeniu każdego ze składników środowiska.
Na system składają się wirtualne efektory i receptory obsługujące odpowiednią maszynę symulacyjną, a także pakiety wspomagające symulację.
Do wykonania zadania użyto programowej struktury ramowej ROS i symulatora Gazebo.

W ramach pracy, stworzone zostały modele dynamiki oraz kinematyki platformy.
Stworzono modele skanerów laserowych, jednostki inercyjnej, oraz enkoderów, a także narzędzia wspomagające testowanie i uruchomienie odpowiednich składników systemu.

Wykonano również szereg testów, mających na celu weryfikację działania opracowanego modelu.

{\centering\Large\bfseries Abstract \\}
\noindent{\bfseries Title: } Simulation of an omnidirectional mobile base \\
\noindent{\bfseries Keywords: } \texttt{simulation}, \texttt{ROS}, \texttt{Gazebo}, \texttt{Mecanum wheels}, \texttt{omnidirectional mobile base}, \texttt{Velma robot}, \texttt{dynamic model}, \texttt{ODE physics engine}

This thesis describes design and implementation of a simulation environment for omnidirectional mobile platform with four Mecanum wheels.
This platform is a mobile base for a two-arm robot Velma.

The platform model and it's sensors models will then support the development of path-planning and movement algorithms, in order to safely test them before running on
hardware.

The main goal of this thesis is to create a dynamic model of the real mobile platform.

To execute the task, ROS framework and Gazebo simulator were used.
Simulation environment consists of dynamic and kinematic models, laser scanner, inertial measurement unit and simulated encoder. 
Also, many other components have been created in order to ease testing, data logging and execution of simulation.

In order to verify the correctness of the implemented model, a set of tests has been performed.
