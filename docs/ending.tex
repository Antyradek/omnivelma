\chapter{Podsumowanie}
\label{sec:ending}

Stworzono modele platform oraz czujników, a także system wielu innych pakietów usprawniających testowanie i sterowanie robotem.

Eksperymenty przeprowadzone na platformie pokazały, że błędy lokalizacji w modelu mogą być nawet większe, niż w rzeczywistym robocie.
Z punktu widzenia testowania programu symulacyjnego to dobrze, gdyż program sterujący przetestowany na symulatorze na pewno poprawnie określi sterowanie
dla robota.
Nie da się jednoznacznie stwierdzić, jak dobrze modelowana jest sama platforma bez przeprowadzenia skomplikowanych testów.
Dodatkowo, niewielkie zmiany parametrów modelu dynamicznego, to jest nadanie mas i inercji ogniwom, powodują inną jazdę i w szczególności inne odczyty 
przyspieszenia w modelu jednostki inercyjnej.

Pakiety pomocnicze i skrypty uruchamiające okazały się przydatne w sterowaniu platformą, jednak niektóre były nadto skomplikowane.
Stanowią one dobrą bazę do późniejszego rozwijania modelu, w celu odpowiedniego ustawienia parametrów, aby jak najdokładniej przybliżyć jej działanie do platformy.
Pozwalają również dokładnie przetestować budowany program sterujący w niemożliwych do zrealizowania przypadkach testowych.

Pisanie pracy pozwoliło na poznanie nowych narzędzi, w szczególności \LaTeX{}a i podobnych.
