\chapter{Podsumowanie}
\label{sec:ending}

Stworzono modele platform oraz czujników, a także system wielu innych pakietów usprawniających testowanie i sterowanie robotem.

Eksperymenty przeprowadzone na modelu i platformie pokazały, że błędy lokalizacji w modelu mogą być nawet większe, niż w rzeczywistym robocie.
Z punktu widzenia testowania programu symulacyjnego jest to przydatne, gdyż program sterujący, przetestowany na symulatorze, na pewno poprawnie określi sterowanie
dla robota o mniejszych błędach ruchu.
Model kinematyki posiada bardzo wiele parametrów działania, nie tylko w kwestii mas ogniw i ich momentów bezwładności, ale
także w kwestii obsługi maszyny symulacyjnej fizyki.
Istnieje kilka sposobów na nadawanie prędkości kątowej kołom.
Aby znaleźć najodpowiedniejszy, należy przeprowadzić czasochłonne badania nad działaniem każdego z nich, również biorąc pod uwagę sposób działania maszyny do symulacji.
Sama maszyna może być modyfikowana w celu lepszego zamodelowania bazy (gdyż ma otwarty kod), lub zastąpiona inną maszyną do symulacji fizyki.

Testy pokazały również, jak wiele różnych i czasami nielogicznych zachowań występuje w modelach i robocie.
Sama poprawna interpretacja wszystkich tych cech wykresów wymaga dogłębnego zbadania działania robota i maszyny do symulacji fizyki.

Ponieważ model posiada uproszczone modele kół Mecanum, niektóre ich cechy (jak na przykład opór obrotowy rolek) nie mogą być zamodelowane
lub też istnieje jakiś nietypowy sposób (na przykład nieznaczna zmiana kierunku wektora siły tarcia) na zamodelowanie takich własności.

Jednostka inercyjna wykrywa własności robota, nieistniejące w modelu, na przykład drgania powstałe przy nagłej zmianie prędkości platformy.
Próby wprowadzania takich własności do modelu mogą nie być możliwe lub też wymagać kolejnych badań, na przykład w kwestii dodania sprężystości do niektórych więzów.
Istnieje także minimalny szum, zależny w dużym stopniu od ustawień parametrów modelu.

Model skanera laserowego jest bardzo prosty w działaniu, zatem jego ewentualny rozwój nie będzie aż tak skomplikowany, jak innych czujników.

Pakiety pomocnicze i skrypty uruchamiające okazały się przydatne w sterowaniu platformą, jednak niektóre były nadto skomplikowane i niepotrzebne.
Pakiet rozdzielania sterowania platformy może być zastąpiony przez kilkukrotne wywołanie wbudowanego pakietu.
Program do ręcznego sterowania jest nadto skomplikowany a i tak nie był używany do sterowania robotem, ani do przeprowadzania testów z rozdziału \ref{sec:tests}.
Stanowią jednak one dobre wspomaganie do późniejszego rozwijania modelu w celu odpowiedniego ustawienia parametrów, aby jak najdokładniej przybliżyć jej działanie do platformy.


